\documentclass[10pt]{extarticle}
\renewcommand{\arraystretch}{1.5}
\renewcommand{\baselinestretch}{1.5}
\usepackage[onehalfspacing]{setspace}
\setlength{\parindent}{0em}
\setlength{\parskip}{0.2em}
\font\myfont=cmr12 at 26pt
\usepackage{anyfontsize}
\usepackage{tabularx}
\usepackage{multirow}
\pagenumbering{arabic} 
\usepackage{soul}
\usepackage{xcolor}
\usepackage[T1]{fontenc}
\renewcommand{\familydefault}{\sfdefault}
\usepackage{blindtext}
\usepackage{titling}
\setlength{\droptitle}{-14em}   % This is your set screw
\usepackage[english]{babel}
\usepackage{graphicx}
\usepackage{float}
\usepackage{eso-pic}
\graphicspath{ {../images/} {../../forecasting/traffic/charts/profiling} {../../forecasting/traffic/charts/preparation} {../../forecasting/inflation-rate/charts/lab5/profiling} {../../forecasting/inflation-rate/charts/lab5/preparation/aggregation} {../../forecasting/inflation-rate/charts/lab5/preparation/differential} }
\usepackage{subcaption}
\usepackage{geometry}
\usepackage{hyperref}
\usepackage[section]{placeins}
\geometry{margin=2cm, bmargin=2cm, tmargin=3cm}

\newcommand{\ctext}[3][RGB]{%
  \begingroup
  \definecolor{hlcolor}{#1}{#2}\sethlcolor{hlcolor}%
  \hl{#3}%
  \endgroup
}


\begin{document}


\AddToShipoutPictureBG*{
\AtPageUpperLeft{
\hspace{19.5cm}
\raisebox{-2.5cm}{\makebox[0pt][r]{\fontsize{36}{1cm}\selectfont DS 2025\par}}}}

\AddToShipoutPictureBG*{
\AtPageUpperLeft{
\hspace{6.5cm}
\raisebox{-3.5cm}{
\makebox[0pt][r]{ 
\includegraphics[scale=0.95]{tecnico_logo.jpg}\\[3cm]}}}}


\title{{\myfont Data Science Lab 5}}  % Title
\setlength{\droptitle}{1cm}

\date{\vspace{-9ex}} % Date for the report, skipped and used to adjust height
\maketitle % Insert the title, author and date
\begin{center}
    %\setlength\extrarowheight{7pt}
    \begin{tabular}{ |l|l l|l| }
        \hline
        \multirow{4}{6em}{\textbf{Team nr:} 2} & \textbf{Student 1:} \r{A}mund Grimstad & \textbf{IST nr:} 1116675\\
        & \textbf{Student 2:} Arthur de Arruda Chau & \textbf{IST nr:} 1116090 \\
        & \textbf{Student 3:} Benjamin Raymond Kuhn & \textbf{IST nr:} 1115778 \\
        & \textbf{Student 4:} João Rafael Freitas Lourenço & \textbf{IST nr:} 425699 \\
        \hline
    \end{tabular}
\end{center}


\begin{center}
	\section*{\fontsize{0.75cm}{1cm}\selectfont TIME SERIES ANALYSIS}
\end{center}

\section*{Traffic Time Series}

\subsection*{DATA PROFILING}

\subsection*{\textit{Data Dimensionality and Granularity}}
%\ctext[RGB]{190,190,190}{May be used to identify the most atomic granularity and two other different granularities to consider.  \textbf{Shall not exceed 500 characters.}}

\begin{figure}[H]
%\centering\includegraphics[width=0.5\textwidth]{flight_best_MLP_best_recall_eval.png}
\caption{Traffic Time Series at the most granular detail}
\end{figure}

\begin{figure}[H]
%\centering\includegraphics[scale=0.95]{}
\caption{Traffic Time Series at the second chosen granularity}
\end{figure}

\begin{figure}[H]
%\centering\includegraphics[scale=0.95]{}
\caption{Traffic Time Series at the third chosen granularity}
\end{figure}

\subsection*{\textit{Data Distribution}}
%\ctext[RGB]{190,190,190}{Shall be used to perform the data analysis at those three different granularities, concerning the series distribution.  \textbf{Shall not exceed 500 characters.}}

\begin{figure}[H]
%\centering\includegraphics[scale=0.95]{}
\caption{Boxplot(s) for Traffic Time Series}
\end{figure}

\begin{figure}[H]
%\centering\includegraphics[scale=0.95]{}
\caption{Histogram(s) for Traffic Time Series}
\end{figure}

\begin{figure}[H]
%\centering\includegraphics[scale=0.95]{}
\caption{Autocorrelation lag-plots for original Traffic Time Series}
\end{figure}

\begin{figure}[H]
%\centering\includegraphics[scale=0.95]{}
\caption{Autocorrelation correlogram for original Traffic Time Series}
\end{figure}

\subsection*{\textit{Data Stationarity}}
%\ctext[RGB]{190,190,190}{Shall be used to perform the data analysis at those three different granularities, concerning the series stationarity.  \textbf{Shall not exceed 300 characters.}}

\begin{figure}[H]
%\centering\includegraphics[scale=0.95]{}
\caption{Components study for Traffic Time Series}
\end{figure}

\begin{figure}[H]
%\centering\includegraphics[scale=0.95]{}
\caption{Stationarity study for Traffic Time Series}
\end{figure}

\section*{DATA TRANSFORMATION}

\subsection*{\textit{Aggregation}}
%\ctext[RGB]{190,190,190}{Shall describe the results of applying three different aggregations over both datasets, and identifying the granularity chosen to proceed.  \textbf{Shall not exceed 300 characters.}}

\begin{figure}[H]
%\centering\includegraphics[scale=0.95]{}
\caption{Forecasting plots after different aggregations on Traffic Time Series}
\end{figure}

\begin{figure}[H]
%\centering\includegraphics[scale=0.95]{}
\caption{Forecasting results after different aggregations on Traffic Time Series}
\end{figure}

\subsection*{\textit{Differentiation}}
%\ctext[RGB]{190,190,190}{Shall describe the results of applying two consecutive differentiation of both datasets, and identifying the best result to proceed.  \textbf{Shall not exceed 300 characters.}}

\begin{figure}[H]
%\centering\includegraphics[scale=0.95]{}
\caption{Forecasting plots after first and second differentiation of Traffic Time Series}
\end{figure}

\begin{figure}[H]
%\centering\includegraphics[scale=0.95]{}
\caption{Forecasting plots after first and second differentiation of Traffic Time Series}
\end{figure}

\subsection*{\textit{Smoothing}}
%\ctext[RGB]{190,190,190}{Shall describe the results of applying different smoothing tecniques to the training set, and identifying the best result to proceed.  \textbf{Shall not exceed 300 characters.}}

\begin{figure}[H]
%\centering\includegraphics[scale=0.95]{}
\caption{Forecasting plots after different smoothing parameterisations on Traffic Time Series}
\end{figure}

\begin{figure}[H]
%\centering\includegraphics[scale=0.95]{}
\caption{Forecasting results after different smoothing parameterisations on Traffic Time Series}
\end{figure}

\subsection*{\textit{Scaling}}
%\ctext[RGB]{190,190,190}{Shall describe the results of applying different scaling techniques to both datasets, and identifying the best result to proceed.  \textbf{Shall not exceed 300 characters.}}

\begin{figure}[H]
%\centering\includegraphics[scale=0.95]{}
\caption{Forecasting plots after different scaling parameterisations on Traffic Time Series}
\end{figure}

\begin{figure}[H]
%\centering\includegraphics[scale=0.95]{}
\caption{Forecasting results after different scaling parameterisations on Traffic Time Series}
\end{figure}




\section*{Inflation Rate Time Series}

\subsection*{DATA PROFILING}

\subsection*{\textit{Data Dimensionality and Granularity}}
%\ctext[RGB]{190,190,190}{May be used to identify the most atomic granularity and two other different granularities to consider.  \textbf{Shall not exceed 500 characters.}}

\begin{figure}[H]
\centering\includegraphics[width=0.9\textwidth]{inflation_dimensionality.png}
\caption{U.S. Inflation Rate dimensionality over time.}
\end{figure}

\begin{figure}[H]
\centering\includegraphics[width=0.9\textwidth]{inflation_granularity.png}
\caption{Inflation Rate Time Series at three different granularities}
\end{figure}

\subsection*{\textit{Data Distribution}}
%\ctext[RGB]{190,190,190}{Shall be used to perform the data analysis at those three different granularities, concerning the series distribution.  \textbf{Shall not exceed 500 characters.}}

\begin{figure}[H]
\centering\includegraphics[width=0.8\textwidth]{inflation_distribution_5number_summary.png}
\caption{Boxplot(s) for Inflation Rate Time Series}
\end{figure}

\begin{figure}[H]
\centering\includegraphics[width=1.0\textwidth]{inflation_distribution_variables_histogram.png}
\caption{Histogram(s) for Inflation Rate Time Series}
\end{figure}

\begin{figure}[H]
\centering\includegraphics[width=0.8\textwidth]{inflation_distribution_autocorrelation_plot.png}
\caption{Autocorrelation lag-plots for original Inflation Rate Time Series}
\end{figure}

\begin{figure}[H]
\centering\includegraphics[width=0.9\textwidth]{inflation_distribution_autocorrelation_study.png}
\caption{Autocorrelation correlogram for original Inflation Rate Time Series}
\end{figure}

\subsection*{\textit{Data Stationarity}}
%\ctext[RGB]{190,190,190}{Shall be used to perform the data analysis at those three different granularities, concerning the series stationarity.  \textbf{Shall not exceed 300 characters.}}

\begin{figure}[H]
\centering\includegraphics[width=0.9\textwidth]{inflation_seasonality_components.png}
\caption{Components study for Inflation Rate Time Series}
\end{figure}

\begin{figure}[H]
\centering\includegraphics[width=0.9\textwidth]{inflation_stationarity_study.png}
\caption{Stationarity study for Inflation Rate Time Series}
\end{figure}

\begin{figure}[H]
\centering\includegraphics[width=0.9\textwidth]{inflation_stationarity_study_bins.png}
\caption{Stationarity study with bins for Inflation Rate Time Series}
\end{figure}

\begin{figure}[H]
\centering\includegraphics[width=0.3\textwidth]{inflation_stationarity_ADF_stats.png}
\caption{Augmented Dickey-Fuller test results for Inflation Rate Time Series}
\end{figure}

\section*{DATA TRANSFORMATION}

\subsection*{\textit{Aggregation}}
%\ctext[RGB]{190,190,190}{Shall describe the results of applying three different aggregations over both datasets, and identifying the granularity chosen to proceed.  \textbf{Shall not exceed 300 characters.}}

\begin{figure}[H]
%\centering\includegraphics[scale=0.95]{}
\caption{Forecasting plots after different aggregations on Inflation Rate Time Series}
\end{figure}

\begin{figure}[H]
%\centering\includegraphics[scale=0.95]{}
\caption{Forecasting results after different aggregations on Inflation Rate Time Series}
\end{figure}

\subsection*{\textit{Differentiation}}
%\ctext[RGB]{190,190,190}{Shall describe the results of applying two consecutive differentiation of both datasets, and identifying the best result to proceed.  \textbf{Shall not exceed 300 characters.}}

\begin{figure}[H]
%\centering\includegraphics[scale=0.95]{}
\caption{Forecasting plots after first and second differentiation of Inflation Rate Time Series}
\end{figure}

\begin{figure}[H]
%\centering\includegraphics[scale=0.95]{}
\caption{Forecasting plots after first and second differentiation of Inflation Rate Time Series}
\end{figure}

\subsection*{\textit{Smoothing}}
%\ctext[RGB]{190,190,190}{Shall describe the results of applying different smoothing tecniques to the training set, and identifying the best result to proceed.  \textbf{Shall not exceed 300 characters.}}

\begin{figure}[H]
%\centering\includegraphics[scale=0.95]{}
\caption{Forecasting plots after different smoothing parameterisations on Inflation Rate Time Series}
\end{figure}

\begin{figure}[H]
%\centering\includegraphics[scale=0.95]{}
\caption{Forecasting results after different smoothing parameterisations on Inflation Rate Time Series}
\end{figure}

\subsection*{\textit{Scaling}}
%\ctext[RGB]{190,190,190}{Shall describe the results of applying different scaling techniques to both datasets, and identifying the best result to proceed.  \textbf{Shall not exceed 300 characters.}}

\begin{figure}[H]
%\centering\includegraphics[scale=0.95]{}
\caption{Forecasting plots after different scaling parameterisations on Inflation Rate Time Series}
\end{figure}

\begin{figure}[H]
%\centering\includegraphics[scale=0.95]{}
\caption{Forecasting results after different scaling parameterisations on Inflation Rate Time Series}
\end{figure}

\end{document}