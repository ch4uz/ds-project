\documentclass[10pt]{extarticle}
\renewcommand{\arraystretch}{1.5}
\renewcommand{\baselinestretch}{1.5}
\usepackage[onehalfspacing]{setspace}
\setlength{\parindent}{0em}
\setlength{\parskip}{0.2em}
\font\myfont=cmr12 at 26pt
\usepackage{anyfontsize}
\usepackage{tabularx}
\usepackage{multirow}
\pagenumbering{arabic} 
\usepackage{soul}
\usepackage{xcolor}
\usepackage[T1]{fontenc}
\renewcommand{\familydefault}{\sfdefault}
\usepackage{blindtext}
\usepackage{titling}
\setlength{\droptitle}{-14em}   % This is your set screw
\usepackage[english]{babel}
\usepackage{graphicx}
\usepackage{float}
\usepackage{eso-pic}
\graphicspath{ {../images/} {../../forecasting/traffic/charts/profiling} {../../forecasting/traffic/charts/preparation} {../../forecasting/inflation-rate/charts/lab5/profiling} {../../forecasting/inflation-rate/charts/lab5/preparation/aggregation} {../../forecasting/inflation-rate/charts/lab5/preparation/differential} {../../forecasting/inflation-rate/charts/lab5/preparation/smoothing} {../../forecasting/inflation-rate/charts/lab5/preparation/scaling}}
\usepackage{subcaption}
\usepackage{geometry}
\usepackage{hyperref}
\usepackage[section]{placeins}
\geometry{margin=2cm, bmargin=2cm, tmargin=3cm}

\newcommand{\ctext}[3][RGB]{%
  \begingroup
  \definecolor{hlcolor}{#1}{#2}\sethlcolor{hlcolor}%
  \hl{#3}%
  \endgroup
}


\begin{document}


\AddToShipoutPictureBG*{
\AtPageUpperLeft{
\hspace{19.5cm}
\raisebox{-2.5cm}{\makebox[0pt][r]{\fontsize{36}{1cm}\selectfont DS 2025\par}}}}

\AddToShipoutPictureBG*{
\AtPageUpperLeft{
\hspace{6.5cm}
\raisebox{-3.5cm}{
\makebox[0pt][r]{ 
\includegraphics[scale=0.95]{tecnico_logo.jpg}\\[3cm]}}}}


\title{{\myfont Data Science Lab 5}}  % Title
\setlength{\droptitle}{1cm}

\date{\vspace{-9ex}} % Date for the report, skipped and used to adjust height
\maketitle % Insert the title, author and date
\begin{center}
    %\setlength\extrarowheight{7pt}
    \begin{tabular}{ |l|l l|l| }
        \hline
        \multirow{4}{6em}{\textbf{Team nr:} 2} & \textbf{Student 1:} \r{A}mund Grimstad & \textbf{IST nr:} 1116675\\
        & \textbf{Student 2:} Arthur de Arruda Chau & \textbf{IST nr:} 1116090 \\
        & \textbf{Student 3:} Benjamin Raymond Kuhn & \textbf{IST nr:} 1115778 \\
        & \textbf{Student 4:} João Rafael Freitas Lourenço & \textbf{IST nr:} 425699 \\
        \hline
    \end{tabular}
\end{center}


\begin{center}
	\section*{\fontsize{0.75cm}{1cm}\selectfont TIME SERIES ANALYSIS}
\end{center}

\section*{Traffic Time Series}

\subsection*{DATA PROFILING}

\subsection*{\textit{Data Dimensionality and Granularity}}
%\ctext[RGB]{190,190,190}{May be used to identify the most atomic granularity and two other different granularities to consider.  \textbf{Shall not exceed 500 characters.}}

\begin{figure}[H]
%\centering\includegraphics[width=0.5\textwidth]{flight_best_MLP_best_recall_eval.png}
\caption{Traffic Time Series at the most granular detail}
\end{figure}

\begin{figure}[H]
%\centering\includegraphics[scale=0.95]{}
\caption{Traffic Time Series at the second chosen granularity}
\end{figure}

\begin{figure}[H]
%\centering\includegraphics[scale=0.95]{}
\caption{Traffic Time Series at the third chosen granularity}
\end{figure}

\subsection*{\textit{Data Distribution}}
%\ctext[RGB]{190,190,190}{Shall be used to perform the data analysis at those three different granularities, concerning the series distribution.  \textbf{Shall not exceed 500 characters.}}

\begin{figure}[H]
%\centering\includegraphics[scale=0.95]{}
\caption{Boxplot(s) for Traffic Time Series}
\end{figure}

\begin{figure}[H]
%\centering\includegraphics[scale=0.95]{}
\caption{Histogram(s) for Traffic Time Series}
\end{figure}

\begin{figure}[H]
%\centering\includegraphics[scale=0.95]{}
\caption{Autocorrelation lag-plots for original Traffic Time Series}
\end{figure}

\begin{figure}[H]
%\centering\includegraphics[scale=0.95]{}
\caption{Autocorrelation correlogram for original Traffic Time Series}
\end{figure}

\subsection*{\textit{Data Stationarity}}
%\ctext[RGB]{190,190,190}{Shall be used to perform the data analysis at those three different granularities, concerning the series stationarity.  \textbf{Shall not exceed 300 characters.}}

\begin{figure}[H]
%\centering\includegraphics[scale=0.95]{}
\caption{Components study for Traffic Time Series}
\end{figure}

\begin{figure}[H]
%\centering\includegraphics[scale=0.95]{}
\caption{Stationarity study for Traffic Time Series}
\end{figure}

\section*{DATA TRANSFORMATION}

\subsection*{\textit{Aggregation}}
%\ctext[RGB]{190,190,190}{Shall describe the results of applying three different aggregations over both datasets, and identifying the granularity chosen to proceed.  \textbf{Shall not exceed 300 characters.}}

\begin{figure}[H]
%\centering\includegraphics[scale=0.95]{}
\caption{Forecasting plots after different aggregations on Traffic Time Series}
\end{figure}

\begin{figure}[H]
%\centering\includegraphics[scale=0.95]{}
\caption{Forecasting results after different aggregations on Traffic Time Series}
\end{figure}

\subsection*{\textit{Differentiation}}
%\ctext[RGB]{190,190,190}{Shall describe the results of applying two consecutive differentiation of both datasets, and identifying the best result to proceed.  \textbf{Shall not exceed 300 characters.}}

\begin{figure}[H]
%\centering\includegraphics[scale=0.95]{}
\caption{Forecasting plots after first and second differentiation of Traffic Time Series}
\end{figure}

\begin{figure}[H]
%\centering\includegraphics[scale=0.95]{}
\caption{Forecasting plots after first and second differentiation of Traffic Time Series}
\end{figure}

\subsection*{\textit{Smoothing}}
%\ctext[RGB]{190,190,190}{Shall describe the results of applying different smoothing tecniques to the training set, and identifying the best result to proceed.  \textbf{Shall not exceed 300 characters.}}

\begin{figure}[H]
%\centering\includegraphics[scale=0.95]{}
\caption{Forecasting plots after different smoothing parameterisations on Traffic Time Series}
\end{figure}

\begin{figure}[H]
%\centering\includegraphics[scale=0.95]{}
\caption{Forecasting results after different smoothing parameterisations on Traffic Time Series}
\end{figure}

\subsection*{\textit{Scaling}}
%\ctext[RGB]{190,190,190}{Shall describe the results of applying different scaling techniques to both datasets, and identifying the best result to proceed.  \textbf{Shall not exceed 300 characters.}}

\begin{figure}[H]
%\centering\includegraphics[scale=0.95]{}
\caption{Forecasting plots after different scaling parameterisations on Traffic Time Series}
\end{figure}

\begin{figure}[H]
%\centering\includegraphics[scale=0.95]{}
\caption{Forecasting results after different scaling parameterisations on Traffic Time Series}
\end{figure}




\section*{Inflation Rate Time Series}

\subsection*{DATA PROFILING}

\subsection*{\textit{Data Dimensionality and Granularity}}
%\ctext[RGB]{190,190,190}{May be used to identify the most atomic granularity and two other different granularities to consider.  \textbf{Shall not exceed 500 characters.}}

\begin{figure}[H]
\centering\includegraphics[width=0.9\textwidth]{inflation_dimensionality.png}
\caption{U.S. Inflation Rate dimensionality over time.}
\end{figure}

\begin{figure}[H]
\centering\includegraphics[width=0.9\textwidth]{inflation_granularity.png}
\caption{Inflation Rate Time Series at three different granularities}
\end{figure}

\subsection*{\textit{Data Distribution}}
%\ctext[RGB]{190,190,190}{Shall be used to perform the data analysis at those three different granularities, concerning the series distribution.  \textbf{Shall not exceed 500 characters.}}

\begin{figure}[H]
\centering\includegraphics[width=0.8\textwidth]{inflation_distribution_5number_summary.png}
\caption{Boxplot(s) for Inflation Rate Time Series}
\end{figure}

\begin{figure}[H]
\centering\includegraphics[width=1.0\textwidth]{inflation_distribution_variables_histogram.png}
\caption{Histogram(s) for Inflation Rate Time Series}
\end{figure}

\begin{figure}[H]
\centering\includegraphics[width=0.8\textwidth]{inflation_distribution_autocorrelation_plot.png}
\caption{Autocorrelation lag-plots for original Inflation Rate Time Series}
\end{figure}

\begin{figure}[H]
\centering\includegraphics[width=0.9\textwidth]{inflation_distribution_autocorrelation_study.png}
\caption{Autocorrelation correlogram for original Inflation Rate Time Series}
\end{figure}

\subsection*{\textit{Data Stationarity}}
%\ctext[RGB]{190,190,190}{Shall be used to perform the data analysis at those three different granularities, concerning the series stationarity.  \textbf{Shall not exceed 300 characters.}}

\begin{figure}[H]
\centering\includegraphics[width=0.9\textwidth]{inflation_seasonality_components.png}
\caption{Components study for Inflation Rate Time Series}
\end{figure}

\begin{figure}[H]
\centering\includegraphics[width=0.9\textwidth]{inflation_stationarity_study.png}
\caption{Stationarity study for Inflation Rate Time Series}
\end{figure}

\begin{figure}[H]
\centering\includegraphics[width=0.9\textwidth]{inflation_stationarity_study_bins.png}
\caption{Stationarity study with bins for Inflation Rate Time Series}
\end{figure}

\begin{figure}[H]
\centering\includegraphics[width=0.3\textwidth]{inflation_stationarity_ADF_stats.png}
\caption{Augmented Dickey-Fuller test results for Inflation Rate Time Series}
\end{figure}

\section*{DATA TRANSFORMATION}

\subsection*{\textit{Aggregation}}
%\ctext[RGB]{190,190,190}{Shall describe the results of applying three different aggregations over both datasets, and identifying the granularity chosen to proceed.  \textbf{Shall not exceed 300 characters.}}

\begin{figure}[H]
\centering\includegraphics[width=0.8\textwidth]{Aggregation_persistence_real_3M_forecast.png}
\centering\includegraphics[width=0.8\textwidth]{Aggregation_linear_regression_3M_forecast.png}
\caption{Forecasting predictions from Persistence Realist and Linear Regression on \textbf{Quarterly
         Inflation Aggregation}}
\end{figure}

\begin{figure}[H]
\centering\includegraphics[width=0.45\textwidth]{Aggregation_persistence_real_3M_eval.png}
\centering\includegraphics[width=0.45\textwidth]{Aggregation_linear_regression_3M_eval.png}
\caption{Performance of Persistence Realist and Linear Regression on \textbf{Quarterly
         Inflation Aggregation}}
\end{figure}

\begin{figure}[H]
\centering\includegraphics[width=0.8\textwidth]{Aggregation_persistence_real_6M_forecast.png}
\centering\includegraphics[width=0.8\textwidth]{Aggregation_linear_regression_6M_forecast.png}
\caption{Forecasting predictions from Persistence Realist and Linear Regression on \textbf{Semi-Annual
         Inflation Aggregation}}
\end{figure}

\begin{figure}[H]
\centering\includegraphics[width=0.45\textwidth]{Aggregation_persistence_real_6M_eval.png}
\centering\includegraphics[width=0.45\textwidth]{Aggregation_linear_regression_6M_eval.png}
\caption{Performance of Persistence Realist and Linear Regression on \textbf{Semi-Annual
         Inflation Aggregation}}
\end{figure}

\begin{figure}[H]
\centering\includegraphics[width=0.8\textwidth]{Aggregation_persistence_real_A_forecast.png}
\centering\includegraphics[width=0.8\textwidth]{Aggregation_linear_regression_A_forecast.png}
\caption{Forecasting predictions from Persistence Realist and Linear Regression on \textbf{Annual
         Inflation Aggregation}}
\end{figure}

\begin{figure}[H]
\centering\includegraphics[width=0.45\textwidth]{Aggregation_persistence_real_A_eval.png}
\centering\includegraphics[width=0.45\textwidth]{Aggregation_linear_regression_A_eval.png}
\caption{Performance of Persistence Realist and Linear Regression on \textbf{Annual
         Inflation Aggregation}}
\end{figure}

For all aggregation techniques, the Linear Regression model yields a lower MAE and RMSE than 
the Persisent Realist Model for the test set. Among the Linear Regression models, the Annual 
aggregation yields the lowest MAE for the test set (1.6).

\subsection*{\textit{Differentiation}}
%\ctext[RGB]{190,190,190}{Shall describe the results of applying two consecutive differentiation of both datasets, and identifying the best result to proceed.  \textbf{Shall not exceed 300 characters.}}

\begin{figure}[H]
\centering\includegraphics[width=0.8\textwidth]{Differentiation_persistence_real_diff1_forecast.png}
\centering\includegraphics[width=0.8\textwidth]{Differentiation_linear_regression_diff1_forecast.png}
\caption{Forecasting predictions from Persistence Realist and Linear Regression on \textbf{Annual
         Inflation Differentiation (1st Order)}}
\end{figure}

\begin{figure}[H]
\centering\includegraphics[width=0.45\textwidth]{Differentiation_persistence_real_diff1_eval.png}
\centering\includegraphics[width=0.45\textwidth]{Differentiation_linear_regression_diff1_eval.png}
\caption{Performance of Persistence Realist and Linear Regression on \textbf{Annual
         Inflation Differentiation (1st Order)}}
\end{figure}

\begin{figure}[H]
\centering\includegraphics[width=0.8\textwidth]{Differentiation_persistence_real_diff2_forecast.png}
\centering\includegraphics[width=0.8\textwidth]{Differentiation_linear_regression_diff2_forecast.png}
\caption{Forecasting predictions from Persistence Realist and Linear Regression on \textbf{Annual
         Inflation Differentiation (2nd Order)}}
\end{figure}

\begin{figure}[H]
\centering\includegraphics[width=0.45\textwidth]{Differentiation_persistence_real_diff2_eval.png}
\centering\includegraphics[width=0.45\textwidth]{Differentiation_linear_regression_diff2_eval.png}
\caption{Performance of Persistence Realist and Linear Regression on \textbf{Annual
         Inflation Differentiation (2nd Order)}}
\end{figure}

\begin{figure}[H]
\centering\includegraphics[width=0.8\textwidth]{Differentiation_persistence_real_diff_lag2_forecast.png}
\centering\includegraphics[width=0.8\textwidth]{Differentiation_linear_regression_diff_lag2_forecast.png}
\caption{Forecasting predictions from Persistence Realist and Linear Regression on \textbf{Annual
         Inflation Differentiation with Lag 2}}
\end{figure}

\begin{figure}[H]
\centering\includegraphics[width=0.45\textwidth]{Differentiation_persistence_real_diff_lag2_eval.png}
\centering\includegraphics[width=0.45\textwidth]{Differentiation_linear_regression_diff_lag2_eval.png}
\caption{Performance of Persistence Realist and Linear Regression on \textbf{Annual
         Inflation Differentiation with Lag 2}}
\end{figure}

for diff\_1 and diff\_2, the Linear Regression Model yields a lower MAE and RMSE for the test set than the 
Persistent Realist models. For the diff\_lag2 treatment, the Persistent Realist has a lower MAE (but 
higher RMSE) than the Linear Regression Model. But overall, the best model, a Linear Regression Model 
trained on the diff\_1 treatment, is worse than the best model without differential treatment. So, we 
do not differentiate.

\subsection*{\textit{Smoothing}}
%\ctext[RGB]{190,190,190}{Shall describe the results of applying different smoothing tecniques to the training set, and identifying the best result to proceed.  \textbf{Shall not exceed 300 characters.}}

\begin{figure}[H]
\centering\includegraphics[width=0.8\textwidth]{Smoothing_persistence_real_smooth_2_forecast.png}
\centering\includegraphics[width=0.8\textwidth]{Smoothing_linear_regression_smooth_2_forecast.png}
\caption{Forecasting predictions from Persistence Realist and Linear Regression on \textbf{Annual
         Inflation Smoothing (windowSize=2)}}
\end{figure}

\begin{figure}[H]
\centering\includegraphics[width=0.45\textwidth]{Smoothing_persistence_real_smooth_2_eval.png}
\centering\includegraphics[width=0.45\textwidth]{Smoothing_linear_regression_smooth_2_eval.png}
\caption{Performance of Persistence Realist and Linear Regression on \textbf{Annual
         Inflation Smoothing (windowSize=2)}}
\end{figure}

\begin{figure}[H]
\centering\includegraphics[width=0.8\textwidth]{Smoothing_persistence_real_smooth_3_forecast.png}
\centering\includegraphics[width=0.8\textwidth]{Smoothing_linear_regression_smooth_3_forecast.png}
\caption{Forecasting predictions from Persistence Realist and Linear Regression on \textbf{Annual
         Inflation Smoothing (windowSize=3)}}
\end{figure}

\begin{figure}[H]
\centering\includegraphics[width=0.45\textwidth]{Smoothing_persistence_real_smooth_3_eval.png}
\centering\includegraphics[width=0.45\textwidth]{Smoothing_linear_regression_smooth_3_eval.png}
\caption{Performance of Persistence Realist and Linear Regression on \textbf{Annual
         Inflation Smoothing (windowSize=3)}}
\end{figure}

\begin{figure}[H]
\centering\includegraphics[width=0.8\textwidth]{Smoothing_persistence_real_smooth_4_forecast.png}
\centering\includegraphics[width=0.8\textwidth]{Smoothing_linear_regression_smooth_4_forecast.png}
\caption{Forecasting predictions from Persistence Realist and Linear Regression on \textbf{Annual
         Inflation Smoothing (windowSize=4)}}
\end{figure}

\begin{figure}[H]
\centering\includegraphics[width=0.45\textwidth]{Smoothing_persistence_real_smooth_4_eval.png}
\centering\includegraphics[width=0.45\textwidth]{Smoothing_linear_regression_smooth_4_eval.png}
\caption{Performance of Persistence Realist and Linear Regression on \textbf{Annual
         Inflation Smoothing (windowSize=4)}}
\end{figure}

Choosing the smoothing with windowSize=2 since the Persistence Scored better than the best after 
aggregation and differentiation.

\subsection*{\textit{Scaling}}
%\ctext[RGB]{190,190,190}{Shall describe the results of applying different scaling techniques to both datasets, and identifying the best result to proceed.  \textbf{Shall not exceed 300 characters.}}

\begin{figure}[H]
\centering\includegraphics[width=0.8\textwidth]{Scaling_persistence_real_RobustScaler_forecast.png}
\centering\includegraphics[width=0.8\textwidth]{Scaling_linear_regression_RobustScaler_forecast.png}
\caption{Forecasting predictions from Persistence Realist and Linear Regression on \textbf{Annual
         Inflation Scaling - RobustScaler}}
\end{figure}

\begin{figure}[H]
\centering\includegraphics[width=0.45\textwidth]{Scaling_persistence_real_RobustScaler_eval.png}
\centering\includegraphics[width=0.45\textwidth]{Scaling_linear_regression_RobustScaler_eval.png}
\caption{Performance of Persistence Realist and Linear Regression on \textbf{Annual
         Inflation Scaling - RobustScaler}}
\end{figure}

\begin{figure}[H]
\centering\includegraphics[width=0.8\textwidth]{Scaling_persistence_real_MinMaxScaler_forecast.png}
\centering\includegraphics[width=0.8\textwidth]{Scaling_linear_regression_MinMaxScaler_forecast.png}
\caption{Forecasting predictions from Persistence Realist and Linear Regression on \textbf{Annual
         Inflation Scaling - MinMaxScaler}}
\end{figure}

\begin{figure}[H]
\centering\includegraphics[width=0.45\textwidth]{Scaling_persistence_real_MinMaxScaler_eval.png}
\centering\includegraphics[width=0.45\textwidth]{Scaling_linear_regression_MinMaxScaler_eval.png}
\caption{Performance of Persistence Realist and Linear Regression on \textbf{Annual
         Inflation Scaling - MinMaxScaler}}
\end{figure}

\begin{figure}[H]
\centering\includegraphics[width=0.8\textwidth]{Scaling_persistence_real_StandardScaler_forecast.png}
\centering\includegraphics[width=0.8\textwidth]{Scaling_linear_regression_StandardScaler_forecast.png}
\caption{Forecasting predictions from Persistence Realist and Linear Regression on \textbf{Annual
         Inflation Scaling - StandardScaler}}
\end{figure}

\begin{figure}[H]
\centering\includegraphics[width=0.45\textwidth]{Scaling_persistence_real_StandardScaler_eval.png}
\centering\includegraphics[width=0.45\textwidth]{Scaling_linear_regression_StandardScaler_eval.png}
\caption{Performance of Persistence Realist and Linear Regression on \textbf{Annual
         Inflation Scaling - StandardScaler}}
\end{figure}

The best results are achieved with MinMaxScaler combined with the Persistence Realist model, due to 
its lowest Test MAE (0.35) and significantly better Test MAPE (0.61).

\end{document}